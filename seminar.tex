\documentclass[11pt]{article}
\usepackage{MyTemplate}
\begin{document}
\thispagestyle{empty}
\settextfont[Scale=1.2]{B Nazanin}
\begin{center}
\includegraphics{logo}
\vskip 1cm
{\bf
دانشگاه صنعتی شریف\\ دانشکده مهندسی کامپیوتر\\ سمینار کارشناسی ارشد گرایش هوش مصنوعی\\
\vskip 1cm
عنوان:\\
بازشناسی ریزدانه‌ای شی‬\\
\lr{‫Fine-grained Object Recognition}
\vskip 1cm
نگارش:\\
یاسر سوری\\
۹۲۲۰۴۷۴۴\\
\vskip 1cm
استاد راهنما:\\
دکتر شهره کسایی\\
\vskip 1cm
استاد ممتحن داخلی:\\
دکتر محمد تقی منظوری شلمانی\\

\vskip 3.5cm

}
شهریور ۹۳
\newpage
\end{center}




\settextfont[Scale=1]{XB Yas}
\setlatintextfont[Scale=0.95]{Times New Roman}
%\settextfont{B Nazanin}
%\settextfont{XB Yas}
%\settextfont{XB Kayhan}

{\bf {چکيده: }}
دسته‌بندی تصاویر ریزدانه‌ای عبارت است از دسته‌بندی تصاویر در حالتی که دسته‌های مورد نظر همگی زیر دسته‌ی یک دسته‌ی کلی‌تر هستند. برای مثال وقتی دسته‌های ما همگی گونه‌های مختلف پرندگان هستند، همگی زیر دسته‌ی پرنده هستند. در این حالت خاص مسئله دسته‌ها از نظر ظاهری معمولاً از نظر ظاهری بسیار به یکدیگر شبیه هستند به گونه‌ای که افراد غیر متخصص نمی‌توانند دسته‌ها را از یکدیگر تمایز دهند. در چنین شرایطی راه حل‌های ارائه شده برای مسئله دسته‌بندی تصاویر معمولی اکثراً نتایج خوبی کسب نمی‌کنند. لذا ارائه روش‌هایی جدید برای حل این مسئله لازم است.
در این گزارش ابتدا به مرور روش‌های مهم در دسته‌بندی تصاویر معمولی و سپس به مرور روش‌های ارائه شده برای دسته‌بندی تصاویر ریزدانه‌ای می‌پردازیم. سپس روش انتخاب شده و دلایل انتخاب آن را بررسی می‌کنیم.
% تعریف مختصر راه حل
% در این گزارش چه مواردی بیان خواهد شد و چه نتیجه‌ای گرفته خواهد شد.


{\bf  { واژه‌های کلیدی: }}
بینایی کامپیوتری، بازشناسی شیء، دسته‌بندی تصاویر، بازشناسی ریزدانه‌ای، دسته‌بندی تصاویر ریزدانه‌ای.

\setlength{\parindent}{0.25in} %The indent of the paragraph first line

\section{مقدمه}\label{intro}
% تعریف مسئله
در دسته‌بندی تصویر\footnote{\lr{Image classification}} هر تصویر با توجه به محتوایش دسته‌بندی می‌شود. برای مثال آیا تصویر شامل شی‌ء خودرو هست یا خیر. معمولاً در بینایی کامپیوتری مسئله بدین صورت است که تعدادی دسته مشخص را در نظر می‌گیریم (مثلاً انسان، خودرو، ساختمان، تلویزیون، صندلی، اسب و ...) سپس طبق چارچوب معمول یادگیری ماشین، توسط تعدادی تصویر شامل یکی از دسته‌ها (نمونه‌های مثبت) و تعدادی تصویر بدون شی‌ای از آن دسته (نمونه‌های منفی) یادگیری برای آن دسته انجام می‌شود. در نهایت پس از یادگیری تمام دسته‌ها در مواجهه با تصویر جدید لازم است تشخیص دهیم که آیا شی‌ای از هر کدام از آن دسته‌های مورد بررسی در تصویر وجود دارد یا خیر
\cite{caltech101}.

% کاربردها
% مراحل مختلف یک ردیاب عمومی


% اهمیت موضوع
مسئله مطرح بوده است.

% چالش‌ها
مسئله در حالت کلی بسیار مشکل است.


یک مسئله معکوس است. 

% دیتا ست و معیارهای مقایسه
برای مقایسه کارایی باشد.


\section{کارهای پیشین}\label{sec2}
به صورت یک مسئله تخمین حالت  سیستم پویا\footnote{\lr{Dynamic System}} در نظر گرفت.
\begin{equation}\label{equ:bayes}
p(X_t|\mathcal{I}_t) \propto  \underbrace{p(I_t|X_t)}_\text{درستنمایی}  \int{ \underbrace{p(X_t|X_{t-1})}_\text{مدل جابجایی} p(X_{t-1}|\mathcal{I}_{t-1}) dX_{t-1}}
\end{equation}

\linespread{1}
\small
\setlength{\parskip}{0pt}
\setlength{\parsep}{0pt}

\renewcommand{\bibname}{مراجع}
\begin{latin}
\bibliographystyle{IEEEtran}
%\bibliography{IEEEfull,ref}
\bibliography{IEEEabrv,ref}
\end{latin}



\section*{واژه‌نامه}
%\begin{LTR}
\begin{multicols}{3}
\theendnotes 
\end{multicols}
%\end{LTR}
\end{document}
