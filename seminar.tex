\documentclass[11pt]{article}
\usepackage{MyTemplate}
\begin{document}
\thispagestyle{empty}
\settextfont[Scale=1.2]{B Nazanin}
\begin{center}
\includegraphics{logo}
\vskip 1cm
{\bf
دانشگاه صنعتی شریف\\ دانشکده مهندسی کامپیوتر\\ سمینار کارشناسی ارشد گرایش هوش مصنوعی\\
\vskip 1cm
عنوان:\\
بازشناسی ریزدانه‌ای شی‬\\
\lr{‫Fine-grained Object Recognition}
\vskip 1cm
نگارش:\\
یاسر سوری\\
۹۲۲۰۴۷۴۴\\
\vskip 1cm
استاد راهنما:\\
دکتر شهره کسایی\\
\vskip 1cm
استاد ممتحن داخلی:\\
دکتر محمد تقی منظوری شلمانی\\

\vskip 3.5cm

}
شهریور ۹۳
\newpage
\end{center}




\settextfont[Scale=1]{XB Yas}
\setlatintextfont[Scale=0.95]{Times New Roman}
%\settextfont{B Nazanin}
%\settextfont{XB Yas}
%\settextfont{XB Kayhan}

{\bf {چکيده: }}
در فولان و بهمان


{\bf  { واژه‌های کلیدی: }}
بینایی کامپیوتری، بازشناسی شیء، دسته‌بندی تصاویر، بازشناسی ریزدانه‌ای.

\setlength{\parindent}{0.25in} %The indent of the paragraph first line

\section{مقدمه}\label{intro}
% تعریف مسئله
% کاربردها
% مراحل مختلف یک ردیاب عمومی


% اهمیت موضوع
مسئله مطرح بوده است.

% چالش‌ها
مسئله در حالت کلی بسیار مشکل است.


یک مسئله معکوس است. 

% دیتا ست و معیارهای مقایسه
برای مقایسه کارایی باشد.


\section{کارهای پیشین}\label{sec2}
به صورت یک مسئله تخمین حالت  سیستم پویا\footnote{\lr{Dynamic System}} در نظر گرفت.
\begin{equation}\label{equ:bayes}
p(X_t|\mathcal{I}_t) \propto  \underbrace{p(I_t|X_t)}_\text{درستنمایی}  \int{ \underbrace{p(X_t|X_{t-1})}_\text{مدل جابجایی} p(X_{t-1}|\mathcal{I}_{t-1}) dX_{t-1}}
\end{equation}

\linespread{1}
\small
\setlength{\parskip}{0pt}
\setlength{\parsep}{0pt}

\renewcommand{\bibname}{مراجع}
\begin{latin}
\bibliographystyle{IEEEtran}
%\bibliography{IEEEfull,ref}
\bibliography{IEEEabrv,ref}
\end{latin}

%\renewcommand{\bibname}{مراجع}
%\bibliographystyle{unsrt-fa} % such as plain-fa
%\bibliography{ref}


\section*{واژه‌نامه}
%\begin{LTR}
\begin{multicols}{3}
\theendnotes 
\end{multicols}
%\end{LTR}
\end{document}
